\documentclass[../thesis.tex]{subfiles}
\begin{document}

\chapter{Code}\label{sec:code}
\section{Transfer Matrix Model}
The following is code to implement a transfer matrix model for calculating absorption and electric field profile within a thin film stack, as described by Pettersson.\supercite{Pettersson1999}

transfer.m
\lstinputlisting[language=Octave]{sections/code/Optical-Model/transfer.m}

\section{Out-Coupling (Power Dissipation)}\label{sec:outCoupling_code}
The following model implements an out-coupling calculation by calculating the power dissipation as a function of the normalized in-plane wavevector, $u$.  
This is an implementation of the method outlined by Furno \textit{et al.}.\supercite{Furno2010,Furno2012}

PowerDissipationModel.m
\lstinputlisting[language=Octave]{sections/code/Optical-Model/PowerDissipationModel.m}


\newpage
\section{Database Keys}
\subsection{materials}

\begin{itemize}
\item commonName : string (all lowercase)
\item formattedName : string (as you would like it displayed)
\item chemicalName : string (long chemical name)
\item allNames : list of all possible names
\item LUMO\_list : [[LUMO (eV) , source (string)],...]
\item HOMO\_list : [[HOMO (eV) , source (string)],...]
\item E\_T\_list : [[E\_T (eV) , source (string)],...]
\item LUMO : float (eV) (preferred value)
\item HOMO : float (eV) (preferred value)
\item E\_T : float (eV) (preferred value)
\item CAS : integer (no dashes)
\item molecularWeight : float (amu)
\item Tg : float (Celcius)
\item meltingTemp : float (Celcius)
\item L\_D : float (nm)
\item L\_D\_list : [[L\_D (nm) , source (string)],...]
\item tau : float (ns)
\item tau\_list : [[tau (ns) , source (string)],...]
\end{itemize}


\subsection{architectures}
\begin{itemize}
\item layers\_list - [layer,...]
\begin{itemize}

\item 'layer':int(layernum),
\item 'material':string,
\item 'materialID' : \_id(material),
\item 'type':string,
\item 'thickness':float,
\item 'concentration':float<1
\end{itemize}
\item eml\_thickness-float
\item doping\_concentration-float<1
\item device\_type - string(SEML/DEML/MEML/AEML/GEML)
\end{itemize}



\subsection{growths}

\begin{itemize}
\item experiment\_label - string
\item grower\_name - string
\item growth\_date - datetime.datetime
\item peak\_eqe - float(\%)
\item device\_area - float(cm$^2$)
\item ito\_pattern - string
\item last\_chamber\_clean - datetime.datetime
\item architectureID - \_id(architecture)
\item notes - string
\item tags - [string,...]
\item spectrums - [spectrum,...]

\begin{itemize}
\item current\_density - float (A/cm$^2$),
\item integration\_time - float (s),
\item waves - [float,...] (nm),
\item intensity - [float,...]
\end{itemize}

\item devices - [stat,...]

\begin{itemize}
\item filename - string,
\item devID - string,
\item substrate - int,
\item device - int,
\item area - float,
\item V\_raw - [float,...],
\item I\_dev\_raw - [float,...] (A),
\item I\_photo\_raw - [float,...] (A),
\item V - [float,...] (V),
\item I\_photo - [float,...] (A),
\item J - [float,...] (mA/cm$^2$),
\item P\_opt - [float,...] (W),
\item EQE - [float,...] (\%),
\item peakEQE - float (\%),
\item lumens - [float,...] (lm),
\item cdm2 - [float,...] (cd/m$^2$),
\item P\_eff - [float,...] (lm/W),
\end{itemize}
\end{itemize}

\subsection{lifetimes}

\textbf{General Info}

\begin{itemize}
\item experiment\_label - string
\item test\_date - datetime.datetime
\item growthID - \_id(growth)
\item architectureID - \_id(architecture)
\item notes - string
\item tags -[string,...]
\item device - int
\item substrate - int
\end{itemize}

\textbf{Test Info}
\begin{itemize}
\item background - float(A)
\item box - int
\item compartment - int
\item current - float(A)
\item luminance - int(cd/m2)
\item powerDensity - int(mW/cm2) // Power density of light source for optical and L+J tests
\item break\_interval - float(min) //break time for laser or current
\item test\_type - str([IL/CL/IC]+'-'+[var\_testDrop])
\item pump\_wavelength - float(nm) // Wavelength of PL pump
\item pump\_type - str // obis laser, led, etc.
\end{itemize}

\textbf{Test Data}
\begin{itemize}
\item ELdatetime - [datetime,...]
\item ELonTime - [float,...]
\item voltage - [float,...]
\item initial\_voltage=float
\item ELdevSignal = [float,...]
\item ELnormDevSignal - [float,...]
\item ELtrust - int (0,1,2) //2 is most trusted
\item PLtrust - int (0,1,2) //2 is most trusted
\item PLdatetime - [datetime,...]
\item PLonTime - [float,...]
\item PLdevSignal = [float,...]
\item PLelDecSignal = [float,...] //el decay at the pl points
\item PLlaserSignal = [float,...]
\item PLnormSignal - [float,...] //normalized to laser signal then peak norm
\item EQEdegPoint - [float,...] //the degradation point for each EQE
\item EQEcurrent - [[float,...],...]
\item EQEvoltage - [[float,...],...]
\item EQEel - [[float,...],...]
\item EQErelativeEQE - [[float,...],...]
\end{itemize}

\textbf{Test specific data}
Single-Carrier + Photodegradation
For this test, "voltage" is measured with no applied light, and "PLdevSignal" is measured with no applied current.
\begin{itemize}
\item VoltageWithLight - [float,...] // Voltage measured while light is applied to device

\item PLWithCurrent - [float,...] // PL measured while current is applied to device

\end{itemize}

\textbf{lifetimes}
\begin{itemize}
\item ELt50 - (float(hours),float(hours)) // exp,fit
\item ELt75 - (float(hours),float(hours)) // exp,fit
\item ELt90 - (float(hours),float(hours)) // exp,fit
\item PLt95 - (float(hours),float(hours)) // exp,fit
\item PLt90 - (float(hours),float(hours)) // exp,fit
\item PLt85 - (float(hours),string(fit/experiment))
\item PLt80 - (float(hours),string(fit/experiment))
\end{itemize}

\subsection{absSpectra}
\begin{itemize}
\item tester: string (all lowercase)
\item notes: string
\item label: string
\item waves: wavelengths in nm (float)
\item abs: absorbance (float) - same length as waves
\item date: datetime
\item type: solution or film (string)
\item composition: [layer,...]
\item layer : dict
\begin{itemize}
\item 'layer':int(layernum),
\item 'material':string,
\item 'materialID' : \_id(material),
\item 'thickness':float,
\item 'concentration':float<1 

\end{itemize}
\end{itemize}

\subsection{excSpectra}
\begin{itemize}
\item tester: string (all lowercase)
\item notes: string
\item label: string
\item inslits: slit opening in mm (float)
\item outslits: slit opening in mm (float)
\item pump : pump wavelength in nm
\item waves: wavelengths in nm (float)
\item rawBackground: background scan [float,...] - same length as waves
\item rawPL: pl scan [float,...] - same length as waves
\item pl: background corrected PL [float,...] - same length as waves
\item date: datetime
\item type: solution or film (string)
\item composition: [layer,...]
\item layer : dict
\begin{itemize}

\item 'layer':int(layernum),
\item 'material':string,
\item 'materialID' : \_id(material),
\item 'thickness':float,
\item 'concentration':float<1 
\end{itemize}
\end{itemize}

\subsection{opticalConstants}
\begin{itemize}
\item substrate - string [glass/quartz/Si]
\item acquisitionDate - datetime
\item acquiredBy - name (string)
\item label - string
\item wavelengths = [float,...]
\begin{itemize}
\item n - [float,...]
\item k - [float,...]
\end{itemize}

\item composition: [layer,...]
\item layer : dict
\begin{itemize}

\item 'layer':int(layernum),
\item 'material':string,
\item 'materialID' : \_id(material),
\item 'thickness':float,
\item 'concentration':float<1 
\end{itemize}

\end{itemize}

\subsection{plSpectra}
\begin{itemize}
\item tester: string (all lowercase)
\item notes: string
\item label: string
\item inslits: slit opening in mm (float)
\item outslits: slit opening in mm (float)
\item pump : pump wavelength in nm
\item waves: wavelengths in nm (float)
\item rawBackground: background scan [float,...] - same length as waves
\item rawPL: pl scan [float,...] - same length as waves
\item pl: background corrected PL [float,...] - same length as waves
\item date: datetime
\item type: solution or film (string)
\item tempK: temperature in Kelvin
\item emission\_type: string [fluorescence \ delayed phosphorescence]
\item composition: [layer,...]
\item layer : dict

\begin{itemize}
\item 'layer':int(layernum),
\item 'material':string,
\item 'materialID' : \_id(material),
\item 'thickness':float,
\item 'concentration':float<1 

\end{itemize}
\end{itemize}



\chapter{Lifetime Box Code}\label{sec:lifetime_code}

The following code outlines our implementation of lifetime setup and can be found on the Holmes Group Github page at \url{https://github.umn.edu/HolmesGroup/lifetimeTesting}.
To implement all boxes, code is organized into separate files.
\textit{box.py} contains all general functions shared between all hardware implementations and is the main driver for lifetime.
Different hardware configurations require different commands in order to control the hardware.
These hardware specific implementation details are located in \textit{keithleyBox.py}, \textit{keithleyBox2.py}, and \textit{keysightBox.py}.
To facilitate hardware rearrangement, configuration files, such as \textit{box1.json} coordinate hardware.
Finally, each piece of hardware is uniquely named in Linux using udev rules, outlined in \textit{85-lifetime.rules}.

%box.py
%\lstinputlisting[language=Python]{sections/code/lifetimeTesting/box.py}

%keithleyBox.py
%\lstinputlisting[language=Python]{sections/code/lifetimeTesting/keithleyBox.py}

%keithleyBox2.py
%\lstinputlisting[language=Python]{sections/code/lifetimeTesting/keithleyBox2.py}

%keysightBox.py
%\lstinputlisting[language=Python]{sections/code/lifetimeTesting/keysightBox.py}

%box1.json
%\lstinputlisting{sections/code/lifetimeTesting/box1.json}

%85-lifetime.rules
%\lstinputlisting{sections/code/lifetimeTesting/85-lifetime.rules}



\ifcsdef{mainfile}{}{\printbibliography}
\end{document}
