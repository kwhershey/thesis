
\documentclass[../thesis.tex]{subfiles}
\begin{document}
\chapter{Single Carrier Device Modeling}\label{sec:polaron_density_measurement}

When trying to understand device behavior, it is often important to investigate single carrier devices to understand one charge species at a time or isolate dynamic processes.
An example of this is measurement of the triplet-polaron quenching rate constant, demonstrated in Chapter \ref{sec:pl_measurements}.
Determination of the polaron density is often critical in order to quantify these results.
This is often done by assuming the device is operating within the space charge limit, in which charges have overcome injection barrier limits and transport through the bulk of the material is the limiting process.\cite{Reineke2007,Erickson2014}
In the space-charge limit, current is most simply described using the Mott-Gurney Law, and can be modified to include various trap states to adapt to different semiconductor properties.\cite{Pope1999}
However, space-charge limited current is really only accurate for device behavior at high voltages for thick devices.  
Often, organic layer stacks of interest feature relatively thin layers, and voltages close to the injection limits.
It can be difficult to identify




\ifcsdef{mainfile}{}{\bibliography{../library}}
\end{document}
