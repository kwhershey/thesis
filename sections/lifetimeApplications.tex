\documentclass[../thesis.tex]{subfiles}
\begin{document}
\chapter{Decoupling Degradation Mechanisms - Application to OLEDs}\label{sec:decoupling_applications}

Chapter \ref{sec:integrated_lifetime} discussed a novel approach for decoupling lifetime.  
The studies in this chapter use this technique and exploit this additional information to better understand device behavior.
This includes discussions of \textcite{Hershey2017} in Section \ref{sec:cbp_host}, \textcite{Bangsund2018} in Section \ref{sec:lifetime_meml}, as well as unpublished work in Section \ref{sec:lifetime_dow}.
The work presented in this chapter was done in collaboration with John Bangsund and Gang Qian.
This chapter is organized into three sections, for these different works.


\section{Ambipolar Host EML Thickness Dependence}\label{sec:cbp_host}

\subsection{Motivation and Experimental}

%\begin{wrapfigure}{r}{.4\textwidth}
\begin{figure}[ht]
\centering
\begin{subfigure}{.17\textheight}
\includegraphics[width=\textwidth]{lifetimeApplications/architecture}
\caption{}
\label{fig:cbp_architecture}
\end{subfigure}
\begin{subfigure}{.2\textheight}
\includegraphics[width=\textwidth]{lifetimeApplications/eqe}
\caption{}
\label{fig:cbp_eqe}
\end{subfigure}
\caption{a. Device architecture, featuring EML thicknesses of X=10,20, and 30 nm.  b. External Quantum Efficiency ($\eta_{EQE}$) for the three architectures.  Operational points for lifetime are shown in symbols.}
\end{figure}


Carbazole materials are archetypical hosts for phosphorescent devices, most common among them being 4,4'-Bis(N-carbazolyly)-1,1'-biphenyl (CBP).\supercite{Han2014,OBrien1999a,Cho2014,Price2015,Adachi2001,Watanabe2007,Holmes2003,Adamovich2003}
These devices have a large range in device lifetime with architecture, and were a prime candidate for investigation with the decoupling method for isolating these lifetime sensitivities.
Devices consisted of a 40-nm-thick hole-injection layer of Plexcore AQ1200 spun-cast on a glass substrate coated with a 150-nm-thick layer of  indium-tin-oxide (ITO), followed by a 30-nm-thick hole-transport layer of N,N' -Bis(naphthalen-1-yl)-N,N' -bis(phenyl)-benzidine (NPD), and an EML of 4,4'-Bis(N-carbazolyl)-1,1'-biphenyl (CBP) doped at 6 vol. % with tris[2-phenylpyridinato-C2,N]iridium(III) (Ir(ppy)3).  The emissive layer was capped with a 10-nm-thick layer of 2,2′,2"-(1,3,5-Benzinetriyl)-tris(1-phenyl-1-H-benzimidazole) (TPBi), and a 30-nm-thick layer of tris-(8-hydroxyquinoline)aluminum (Alq3). The cathode for each device consisted of a 0.5-nm-thick layer of LiF and a 100-nm-thick layer of Al.
All devices were fabricated according to the processes outlined in Chapter \ref{sec:oleds_fabrication}.
The structure shown in Figure \ref{fig:cbp_architecture} was used with EML thicknesses of X=10, 20, and 30 nm.
The \eqe for all three EML thicknesses, shown in Figure \ref{fig:cbp_eqe}, shows maximum efficiencies of 15.7\%, 15.3\%, and 15.7\% for EML thicknesses of 10, 20, 30 nm, respectively.
Though similar in peak value, as thicknesses increases, the peak \eqe shifts to higher current.
This is indicative of a shift in the charge dynamics and possible location of the recombination zone (RZ).
Recombination zone has been previously linked with lifetime, with a longer lifetime expected for thicker RZ.\supercite{Zhang2014,Wu2016,Chin2005,Lee2006,Chwang2002,Han2016,Lee2005a,Brown2004,Choong2000,Liu2004}
For a well balanced device, one would expect that expanding the EML would result in a wider RZ and thus longer lifetime.

\subsection{Device Stability}

\begin{figure}[ht]
\centering
\includegraphics[width=\textwidth]{lifetimeApplications/elpl}
\caption{Device decay curves for multiple values of the initial luminance as a function of emissive layer thickness.  Loss in (a) electroluminescence (EL) and (b) photoluminescence (PL) are shown and decrease monotonically with increasing luminance.  For devices with a 10-nm-thick emissive layer, initial luminance values are 1000 $cd/m^2$, 5000 $cd/m^2$, and 7000 $cd/m^2$.  For devices with a 20-nm- or 30-nm-thick emissive layer, initial luminance values are 1000 $cd/m^2$, 5000 $cd/m^2$, and 7100 $cd/m^2$. }
\label{fig:cbp_elpl}
\end{figure}

Figure \ref{fig:cbp_elpl}a shows the conventional EL lifetimes of these devices, and indeed the 30 nm EML shows a longer lifetime than the 10 and 20 nm.  
In Figure \ref{fig:cbp_elpl}b, the intermittent \pl measurements can be seen, discussed in Chapter \ref{sec:integrated_lifetime}, showing the advantage of this technique and the large amount of additional data that is available.
The operational conditions of each device is shown in Table \ref{tab:lifetime_summary}.
Similar currents are used across EML thicknesses, but voltage increases slightly with thickness, as might be expected.
No birefringence was seen under cross polarization after degradation, and no new emission features were observed with degradation, verifying that \oc and $\chi$ remain unchanged during degradation, as discussed in Chapter \ref{sec:integrated_lifetime}.


\begin{wraptable}{r}{.6\textwidth}%[hb]
\centering
\begin{tabular}{c|c|c|c|c}
$d_{EML}$ (nm) & $L_0$ (cd/m$^2$) & $J$ (mA/cm$^2$) & $V_0$ (V) & $t_{50}$ (hours) \\
\hline
& 1000 & 2.2 & 4.2 & 139.0 \\
& 3000 & 7.2 & 5.1 & 39.9 \\
10 & 5000 & 13.6 & 5.4 & 15.8 \\
& 7000 & 14.4 & 6.2 & 6.9 \\
& 9000 & 28.0 &  6.3 & 5.3 \\
\hline
& 1000 & 2.2 & 5.4 & 141.1 \\
& 3000 & 7.2 & 6.0 & 33.1 \\
20 & 5000 & 12.4 & 7.2 & 17.2 \\
& 7100 & 19.2 & 7.3 & 10.0 \\
& 9000 & 24.0 &  7.5 & 8.0 \\
\hline
& 1000 & 2.2 & 5.9 & 474 \\
30 & 5000 & 13.6 & 7.3 & 74.4\\
& 7100 & 19.6 & 7.6 & 46 \\
& 8000 & 22.4 &  7.7 & 38.1 \\

\end{tabular}
\caption{Summary of device lifetimes.  For each device, the starting luminance ($L_0$), current density ($J$), starting voltage ($V_0$) and time at which 50\% of the initial luminance is reached ($t_{50}$) are reported.}
\label{tab:lifetime_summary}
\end{wraptable}


The lifetime decreases with luminance for all architectures.  
While a reduction in the EL lifetime is observed in Figures \ref{fig:cbp_elpl} and \ref{fig:tx_components} in reducing device thickness from 30 nm to 20 nm, little difference is seen between devices having EML thicknesses of 20 nm and 10 nm.  
The degradation in the PL intensity does not appear to be a strong function of EML thickness.  
Indeed, comparing the EL and PL lifetimes with the extracted degradation in exciton formation shows that the EL decay is dominated by a loss in the efficiency of exciton formation with \ef reaching 60\% of its initial value by the time EL has reached 50\%.  
A substantial component of this decay is likely due to non-radiative recombination center formation.\supercite{Kondakov2003,Kondakov2007d}
Over this same period, the PL intensity has only degraded by ~10\% of its initial value.

The similarity in PL degradation observed across all EML thicknesses suggests that the exciton and polaron densities are similar between these devices,\supercite{Giebink2008a,Coburn2017,Lee2017} and that they have similar exciton recombination zone widths.  
The accelerated degradation in the exciton formation efficiency (\ef) observed for devices with EML thicknesses of 10 nm and 20 nm suggests that the recombination zone samples the EML/TPBi interface, which has been previously shown to cause degradation.\supercite{Wang2013,Wang2014}
This change in recombination zone location is also suggested by the \eqe behavior shifting peak location, shown in Figure \ref{fig:cbp_eqe}
To validate this suggestion, the position of the recombination zone was assessed in the devices with EML thickness of 20 and 30 nm devices using a quenching TPTBP sensitizer.  
The position of the recombination zone can be inferred by the corresponding reduction in device \eqe due to quenching by TPTBP,\supercite{Erickson2013a} as discussed in Chapter \ref{sec:rz_measurement}.
The sensitized 30-nm-thick EML devices showed no quenching, suggesting no recombination near the interface, while devices with a 20-nm-thick EML showed quenching only at the EML/TPBi interface, confirming the position of the recombination zone at that interface.  
Devices with a 10-nm-thick EML exhibited changing current-voltage behavior when sensitized with TPTBP, and thus PtTPTBP, an emissive sensitizer with a peak wavelength of 770 nm, was used in 2-nm-thick strips on either side of the EML at 0.5 vol. \%.
This configuration was able to match the current-voltage behavior of the control device while permitting the measurement of emission from PtTPTBP.  
\begin{wrapfigure}{r}{.5\textwidth}
\centering
\includegraphics[width=0.5\textwidth]{lifetimeApplications/tx_components}
\caption{Extracted lifetimes for all 3 architectures as a function of luminance. (a) EL $t_{50}$ (b) PL $t_{90}$ and (c) \ef $t_{60}$.}
\label{fig:tx_components}
\end{wrapfigure}
For devices with a 10-nm-thick EML, strong emission from PtTPTBP is observed from the EML/TPBi interface and weak emission seen from the EML/NPD interface. 
These quenching experiments are shown in Figure \ref{fig:cbp_rz}.
These results suggest that for devices with an EML thickness of 10 nm or 20 nm, the recombination zone samples the EML/TPBi interface, accelerating exciton formation loss.  
While detailed analysis of the relevant degradation mechanism is the subject of future work, previous work has suggested a role played by exciton-polaron interactions.\supercite{Wang2015a,Zhang2016,Giebink2008a,Kondakov2007d,Kondakov2003}

\begin{figure}[ht]
\centering
\includegraphics[width=.8\textwidth]{lifetimeApplications/cbp_rz}
\caption{(a) \eqe as a function of current denisty for TPTBP quenched 30 nm devices.  No quenching is observed in the peak \eqe.  (b) \eqe as a function of current density for TPTBP quenched 20 nm devices.  Quenching is observed for the ETL side quencher, and minimally for the HTL side. (c) EL spectra for PtTPTBP quenched 10 nm EML devices. Emission from the sensitizer only at the ETL. (d) Summary of recombination zone measurements. The 30 nm device shows an RZ that is centered, while the 20 and 10 nm devices have an RZ peaked at the TPBi interface.}
\label{fig:cbp_rz}
\end{figure}

\subsection{Conclusion}

In summary, this work presents a method for decoupling optical and electrical losses during OLED operational decay by attributing the overall reduction in electroluminescence to a loss in \pl or the exciton formation efficiency through \ef.  
Model devices are shown as a function of luminance, with a loss in \ef shown to be the limiting factor for the short-lived devices.  
By measuring the RZ, these devices are shown to be subject to interfacial degradation, only seen in narrow EML devices.
Contrary to the expectation, the RZ is not found to expand with the EML thickness, but rather to shift within the device.

The behavior of \ef and \pl with thickness depends on the architecture system.
In fact, the remaining sections of this chapter will offer alternative cases, showing differing behavior.


\newpage

\clearpage

\section{Mixed Host OLED Luminance Scaling}\label{sec:lifetime_meml}
\begin{wrapfigure}{r}{.4\textwidth}
\centering
\includegraphics[width=.4\textwidth]{lifetimeApplications/meml_eqe}
\caption{(a) Current Density and (b) Luminance as a function of Voltage.  (c) \eqe for all three EML thicknesses.  Inset is M-EML device architecture.}
\label{fig:meml_eqe}
\end{wrapfigure}

\subsection{Motivation}

As discussed in Section \ref{sec:cbp_host}, RZ width has been extensively connected with lifetime, mediated by the exciton and polaron populations.\supercite{Scholz2015,Giebink2008a,Giebink2009a,So2010,Zhang2016,Schmidbauer2013,Wu2016,Lee2006,Chwang2002}
However, despite this observed trend with RZ thickness, the specific role of the RZ in degradation kinetics is still an active area of investigation.
Using a mixed emissive layer (M-EML) architecture, the work described in this section seeks to provide a more concrete connection between the RZ and degradation within the same system.  

\subsection{Experimental}


Devices consisted of a 60-nm-thick hole injection layer (HIL) of poly(thiophene-3-[2[(2-methoxyethoxy)ethoxy]-2,5-diyl)(AQ1200, Sigma Aldrich), a 4,4',4"-tris(N-carbazolyl) triphenylamine (TCTA, TCI America) hole-transport layer (HTL), a mixed-host emissive layer (M-EML) consisting of a 47.5 vol.\% TCTA, 47.5 vol.\% 2,2',2''(1,3,5-benzenetriyl) tris-(1-phenyl-1H-benzimidazole) (TPBi, Lumtec) and 5 vol.\% of the green phosphorescent emitter, \irppy , a TPBi electron-transport layer (ETL), and a LiF (1 nm)/Al (100 nm) cathode. 
All layers were deposited according to the procedures outlined in Section \ref{sec:oleds_fabrication}.
When varying the M-EML thickness (10 nm, 30 nm, 60 nm), the HTL and ETL thicknesses are varied equally to maintain a total device thickness of 100 nm.
Device characteristics are shown in Figure \ref{fig:meml_eqe}, with the efficiency increasing slightly from 17\% to 19\% as the EML thickness increases from 10 to 60 nm.

This device architecture system, shown in the inset of Figure \ref{fig:meml_eqe}, was chosen because of its broad RZ, which spans the entire EML.\supercite{Erickson2013a}
Because of this property, the M-EML thicknes, $d_{EML}$ can be taken as a proxy for the RZ width, and the exciton density can be controlled by modifying the EML.
The increase in RZ thickness is evidenced by the change in onset of the roll-off with increasing RZ width, seen in Figure \ref{fig:meml_eqe}c.

\begin{wrapfigure}{r}{.4\textwidth}
\centering
\includegraphics[width=.4\textwidth]{lifetimeApplications/meml_rz}
\caption{(a) Raw spectra of sensitized devices.  (b) Out-coupling efficiency for \irppy and PtTPTBP across the EML as well as electric field profile.  (c) RZ as a function of current density.  For all currents, the RZ is found to span the entire EML.}
\label{fig:meml_rz}
\end{wrapfigure}

To experimentally confirm the RZ breadth, the 60 nm EML architecture was investigated using PtTPTBP as a sensitizer, using the methodology outlined in Chapter \ref{sec:rz_measurement}.
The exciton density is found to remain above 60\% of the peak across the entire 60 nm M-EML at a current density of 10 mA/cm$^2$, shown in Figure \ref{fig:meml_rz}. 
As current density increases from 0.1 mA/cm$^2$ to 10 mA/cm$^2$, the peak of the RZ migrates from the ETL side to the HTL side of the M-EML. 
These findings are consistent with other reports for similar device architectures,\supercite{Erickson2013a} and confirm that $d_{EM}$ is a good proxy for RZ width.
The 60 nm EML is the thickest investigated EML thickness and should be subject to the most variation in RZ intensity accross the EML.  
Therefore, thinner EML devices are also assumed to have a RZ spanning the EML.

\subsection{Device Stability}


The degradation of these devices at a initial luminance of $L_0$=3,000 cd/m$^2$ is shown in Figure \ref{fig:meml_lifetime}a.
The EL lifetime increases by approximately a factor of 3 in increasing the thickness from 10 nm to 60 nm, and nearly all this enhancement can be attributed to a reduced rate of PL degradation, shown in Figure \ref{fig:meml_lifetime}b.
\begin{wrapfigure}{r}{.4\textwidth}
\centering
\includegraphics[width=.4\textwidth]{lifetimeApplications/meml_lifetime}
\caption{(a) EL lifetime at 3,000 cd/m$^2$ for EML thicknesses of 10,30,60 nm.  (b) The corresponding \pl and \ef degradation.}
\label{fig:meml_lifetime}
\end{wrapfigure}
No trend with thickness is apparent in the \ef decays, which are all within typical device-to-device variation. 
In contrast, the PL decays show a dramatic separation with thickness. 
We also note that a reduction in \ef dominates the overall degradation rate in the 30 nm and 60 nm thick M-EML devices, but is comparable to PL losses in the 10 nm M-EML device. 
These results suggest that reduced degradation in emissive layer PL efficiency may be the primary reason for enhanced stability in M-EML architectures, as compared to with their single-host counterparts. 
Moreover, the combination of improved efficiency roll-off and PL lifetime with an increased RZ width, and thus decreased exciton density, provides further evidence of a link between bimolecular annihilation events and the degradation of PL efficiency. \supercite{Schmidbauer2013}
Losses in \ef, however, appear to be relatively insensitive to exciton density.


To show that exciton density and PL loss are intimately related, the exciton density was matched between architectures by scaling the luminance to the EML thickness ratio.  
The 10, 30, and 60 nm EML devices were operated at initial luminances of 1,000, 3,000, and 6,000 cd/m$^2$, respectively.
The results of this aging are shown in Figure \ref{fig:meml_scaled_lifetime}.
PL degradation is nearly identical for 10, 30, and 60 nm M-EML devices operated at luminances of 1,000, 3,000, and 6,000 cd/m$^2$, respectively. 
Exciton formation efficiency losses, in contrast, are rapidly accelerated as luminance is increased. 
At long times, the PL degradation slows slightly with increasing M-EML thickness, and this is attributed to large differences in exciton formation efficiency losses. 
The exciton density does not remain matched over the course of the entire test due to differences in \ef losses, and consequently the formation rate for exciton quenchers is reduced at long times in thicker M-EML devices. 
This observation of matched PL losses under scaled luminance has been reproduced under a range of scaled luminances from 330 cd/m$^2$ to 15,000 cd/m$^2$, showing the same trend. 
Despite comparable exciton densities in the emissive layer, exciton formation efficiency losses differ substantially, and appear to scale with increased luminance and current density. 
Increased current density would result in a larger polaron density in the transport layers and could lead to an increase in the rate of defect formation mediated by unstable cationic or anionic molecules.
Alternatively, the trend with luminance could be explained as an increase in interfacial photodegradation of the cathode or anode due to device electroluminescence.\supercite{Wang2012,Wang2010a}



An alternative approach to investigating this connection is to look at the scaling relationships with luminance and exciton density.
OLED lifetime has been widely observed to follow a $1/L_0^n$ relationship,42 where $L_0$ is the initial luminance, and $n$ is a device specific parameter typically between 1-2. 
For these devices, $n = 1.8 \pm 0.1$ for the $t_{50}$ of EL and is independent of M-EML thickness. 
As shown in Figure \ref{fig:meml_scaling}a, the degradation in \pl and \ef for a 60 nm M-EML show similar acceleration behavior as a function of luminance, with $n = 1.8$ and $n = 1.75$, respectively. 
Comparable slopes are seen for 10 nm and 30 nm M-EML devices. 

\begin{wrapfigure}{r}{.4\textwidth}
\centering
\includegraphics[width=.4\textwidth]{lifetimeApplications/meml_scaled_lifetime}
\caption{Lifetimes of devices with luminance scaled to match the EML thickness. PL collapses due to matched exciton density.}
\label{fig:meml_scaled_lifetime}
\end{wrapfigure}
However, when scaled by $1/d_{EML}$, as displayed in Figure \ref{fig:meml_scaling}b, \ef and \pl show distinct scaling behavior. 
While PL $t_{85}$ shows a slope of $n = 1.9\pm0.3$, almost identical to the slope under luminance acceleration, \ef $t_{85}$ shows a much shallower slope of $n = 0.5\pm0.2$ (decreasing to $n = 0.3\pm0.3$ at 10,000 cd/m$^2$). 
This raises several important implications.
First, the identical slopes for PL provide further evidence that PL losses in this system are determined by the exciton density and the width of the RZ, and imply that there is a direct scaling law between RZ width and PL lifetime. 
While polaron density can play a role in PL degradation as well, it is unlikely that polaron density scales identically with both luminance and $d_{EML}$, implying that a single exciton driven or an exciton-exciton annihilation driven degradation mechanism is dominant in this system.

\begin{wrapfigure}{r}{.6\textwidth}
\centering
\includegraphics[width=.6\textwidth]{lifetimeApplications/meml_luminance}
\caption{Scaling behavior of \pl and \ef as a function of (a) luminance and (b) exciton density.}
\label{fig:meml_scaling}
\end{wrapfigure}

Second, the shallow dependence of \ef $t_{85}$ on RZ width (and hence exciton density) shown in Figure \ref{fig:meml_scaling}b suggests that excitons play a lesser role in \ef degradation. 
Notably, the difference in scaling with $L_0$ and $d_{EML}$ for \ef $t_{85}$ suggests that multiple degradation mechanisms comprise the total \ef loss.  
Exciton formation loss is often attributed to the accumulation of non-radiative recombination centers in the emissive layer,\supercite{Kondakov2003,Kondakov2007d} and has been linked to exciton-polaron interactions.\supercite{Zhang2017a}
The shallow dependence on RZ width (and hence exciton density) shown in Figure \ref{fig:meml_scaling}b suggests that the $d_{EML}$-dependent increase in \ef degradation to reflects the generation of non-radiative recombination centers by an exciton-mediated process, consistent with these reports. 
However, this mechanism alone cannot fully account for degradation in \ef, as the slope against initial luminancescaling with $L_0$ is much steeper (Figure \ref{fig:meml_scaling}a). 
This contrasting behavior suggests that a second mechanism which is independent of emissive layer exciton density governs \ef losses. 
This behavior is consistent with degradation mediated primarily by polarons or photodegradation of the cathode or anode interface, and may originate outside of the emissive layer. \supercite{Wang2012,Wang2010a}

These findings have implications for efforts in modeling OLED lifetime. 
Most modeling approaches assume that the same defect population responsible for exciton quenching was also responsible for non-radiative recombination of charge carriers. 
This immplies that the quenching population resides entirely in the emissive layer.\supercite{Giebink2008a,Zhang2014}
Defect populations external to the emissive layer have been considered, but only for the purposes of fitting voltage rise.\supercite{Lee2017}
In all cases, the generation of defects is proposed to proceed via bimolecular quenching processes.
While these treatments have yielded reasonable fits of the overall degradation behavior, they are unable to capture the behavior observed here.
Exciton formation and PL degradation would be expected to trend together within these formalisms, whereas Figure \ref{fig:meml_scaling} shows clearly distinct scaling behavior. 
Our results thus show that losses to \pl and \ef likely originate from kinetically distinct mechanisms.  
Moreover, the weak dependence of \ef on exciton density indicates and that degradation defects external to the emissive layer may play an important role in luminance loss, and should be considered in future modeling attempts.
Non-radiative recombination centers could have suitable energetics to serve as exciton quenchers, and vice versa.
However, because losses in \ef and \pl show different dependences with initial luminance and M-EML thickness, the exciton quenchers formed in the EML are likely inefficient non-radiative recombination centers for charge carriers.

\subsection{Conclusion}

In conclusion, we find that broadening the RZ sharply reduces the rate of PL degradation, showing a similar scaling relationship as with initial luminance variation. 
This confirms that PL degradation is strongly dependent on exciton density and has minimal dependence on changes in the polaron density as driven by the RZ.  
However, losses in the exciton formation efficiency (\ef) show a weaker dependence on RZ width, suggesting that \ef losses are less sensitive to exciton density and may partly originate outside of the M-EML in this system. 
Notably, the different dependences of PL and exciton formation efficiency loss on RZ width provide clear evidence that kinetically distinct pathways drive OLED degradation, and that a single degradation mechanism cannot be assumed when attempting to model device lifetime. 
These results highlight the capability of decoupled measurements of \pl and \ef losses to yield useful diagnostic insight into the source of device instability and shed light on the kinetics of degradation and the nature of defects.



\newpage


\section{Application to Commercial Co-Host System}\label{sec:lifetime_dow}
\subsection{Motivation}

In addition to my collaborator John Bangsund, I would also like to acknowledge contributions from from authors at The Dow Chemical Company, Dominea Rathwell, Peter Trefonas, Hong-Yeop Na, and Jeong-HwanJeon.

The previous two sections present two alternaitive machanisms for the dominant degradation pathway within devices, being interfacial and exciton density driven degradation.
Both of these mechanisms are common explanations for device behavior in the literature.
The rate of OLED degradation is widely regarded to depend strongly on the exciton density via exciton-exciton and exciton-polaron processes.\supercite{Scholz2015,Schmidbauer2013a,Giebink2008a,Bangsund2018,Liu2004,So2010} 
The width of the RZ determines the exciton density for a given luminance, and hence sets the rate of emissive layer degradation.\supercite{Bangsund2018,Zhang2014,Chwang2002,Wu2016} 
In other works, the absolute position of the RZ is shown to be significant, as proximity to a transport layer interface can exacerbate degradation,\supercite{Hershey2017,Wang2013,Jeon2015} or can influence exciton confinement and charge balance.\supercite{Coburn2017,Coburn2016a} 
In studies showing interfacial degradation, the RZ is typically heavily pinned at the degrading interface. 
These two mechanisms are often view as seperate issues: for devices with pinned RZs, interfacial degradation is assumed dominant, but for well distributed RZs, broader is viewed as better.


This study presents a unique counter example, in which for commercial materials, both types of degradation are shown to be influential as a function of host material concentration.
Using a co-host system, we find that lifetime can be significantly improved compared to single-host devices despite a reduction in RZ width. 
Most other works on mixed host emissive layers have employed exciplex-forming pairs which jointly show more balanced electron and hole mobilities than either host alone, leading to a broad RZ which spans the majority of the EML.\supercite{Kim2017,Kim2017,Chwang2002,Erickson2011,Song2017}
In this work, an ambipolar host, which features a broad RZ, and a wide energy gap host are used. 
Measurements of the RZ confirm that the mixed host decreases the RZ width and shifts the RZ away from the hole transport layer (HTL). 
The increased degradation rate due to a higher exciton density is offset by reduced degradation at the HTL/emissive layer (EML) interface, leading to an overall enhancement in device lifetime. 

\begin{figure}[ht]
\centering
\includegraphics[width=0.8\textwidth]{lifetimeApplications/dow_iv}
\caption{(a) \eqe for selected host compositions. (b) Peak efficiency and turn on voltage as a function of host composition (\% B). (c) and (d) show current density and luminance as a function of voltage. (e) Material energetics and layer thicknesses.}
\label{fig:dow_iv}
\end{figure}

\subsection{Experimental}
Two proprietary host materials, designated Host A and Host B, were provided by The Dow Chemical Company. 
In commercial devices, a mixture of these hosts was found to yield improved lifetimes up to =8.6 hrs at 15,000 cd/m$^2$ compared to 1.3 hrs and 4.2 hrs for Hosts A and B, respectively, where $t_{90}$ is the time to degrade to 90\% of the initial luminance. 
A hole-injection layer of poly(thiophene-3-[2[(2-methoxyethoxy)ethoxy]-2,5-diyl) (AQ1250) was spin-cast on the ITO anode, followed by a hole-transport layer (HTL) of 4,4',4"-tris(N-carbazolyl)triphenylamine (TCTA). 
The emissive layer (EML) consists of Host A, Host B, or a mixture of the two hosts, and a constant emitter doping concentration of 15 vol. \% fac-tris(2-phenylpyridine)iridium(III) (\irppy). 
The Host A and Host B mixture was varied with compositions of 0\%, 5\%, 15\%, 30\%, 50\%, 70\%, 85\%, and 100\% Host B. 
An electron-transport layer (ETL) of tris-(1-phenyl-1H-benzimidazole) (TPBi) is deposited over the EML, followed by a LiF/Al cathode. 

\begin{wrapfigure}{r}{.6\textwidth}
\centering
\includegraphics[width=0.6\textwidth]{lifetimeApplications/dow_pl}
\caption{(a) Fluorescence and low temperature (10 K) Phosphorescence for both hosts. (b) Optical constant $k$ for both hosts.}
\label{fig:dow_pl}
\end{wrapfigure}
The external quantum efficiency with varying host composition is shown in Figure \ref{fig:dow_iv}a-b, with devices with less than 30\% Host B showing a reduced efficiency and the remaining devices exhibiting efficiencies between 18\% and 19\%. The turn-on voltage to achieve 1 cd/m$^2$, shown in Figure \ref{fig:dow_iv}c, is reduced by over one volt when increasing Host B concentration from 0 to 30\%, then remains constant. 
This behavior is attributed to a change in the injection barrier for electrons into the EML. 

\begin{wrapfigure}{r}{.5\textwidth}
\centering
\includegraphics[width=0.5\textwidth]{lifetimeApplications/dow_lifetime}
\caption{(a) Overal EL and PL loss for selected architectures.  (b) Extracted lifetimes as a function of concentration.}
\label{fig:dow_lifetime}
\end{wrapfigure}

Figure \ref{fig:dow_iv}c shows molecular energy levels of interest for this work, with HOMO levels for Host A and Host B calculated from DFT, band gaps extracted from optical constants obtained by ellipsometry, and triplet energies obtained from low temperature phosphorescence.  
The measurement of triplet energies is further discussed in Appendix \ref{sec:triplets}.
Host B is a material with a relatively narrow optical bandgap of 2.9 eV, a calculated HOMO of 5.3 eV, and a measured triplet energy of 2.6 eV. 
Host A as an optical gap of 3.4 eV, a calculate HOMO of 5.7 eV, and a measured triplet energy of 2.6 eV. 
Despite having a difference in energy gap and fluorescence energy, Host A and Host B have similar triplet energies as extracted from phosphorescence, shown in Figure \ref{fig:dow_pl}.
It is worth noting that the energy levels of Host B reside entirely within those of Host A, and Host A and B do not form an exciplex, as confirmed from measurements of PL on a mixed film and of electroluminescence of a simple bilayer device (ITO/Host A/Host B/LiF/Al), in contrast to the majority of reported co-host architectures. 


\subsection{Results}
Figure \ref{fig:dow_lifetime} demonstrates that the mixed host architectures show the longest lifetimes. 
Host A has the shortest lifetime ( hours), with lifetime rapidly increasing with the addition of small concentrations of Host B. 
A maximum lifetime is seen for 30\% Host B in Host A, after which a steady decline in lifetime is seen as concentration is increased to pure Host B ( hours). 
The simultaneous photoluminescence (PL) lifetimes of these devices is reported in Figure \ref{fig:dow_lifetime}. 
Interestingly, the PL stability increases monotonically as percentage of Host B increases. 
This inverse trend between the PL and EL lifetimes was not observed in the data of Sections \ref{sec:cbp_host} or \ref{sec:lifetime_meml}and suggests that RZ width is maximized at pure Host B, but that maximizing RZ width does not optimize lifetime.


\begin{figure}[ht]
\centering
\includegraphics[width=0.8\textwidth]{lifetimeApplications/dow_rz}
\caption{(a) Hole and electron only devices featuring hosts A and B, as well as a 1:1 mixture.  The related architectures are shown in (c) and (d) for holes and electrons, respectively.  (b) Measured RZs at 2 mA/cm$^2$.}
\label{fig:dow_rz}
\end{figure}


In conventional mixed host architectures, hole- and electron-transporting hosts are blended to improve the balance of electron and hole mobilities, resulting in a broadened RZ and reduced exciton density.\supercite{Chwang2002,Erickson2013a,Han2016,Kondakova2008a}
To investigate how mixing Host A and Host B impacts charge transport, single-carrier devices were fabricated for emissive layers containing 0\%, 50\%, and 100\% Host B. 
The current-voltage characteristics for HOD and EOD are shown in Figure \ref{fig:dow_rz}a-b, along with the device architectures.
For both holes and electrons, Host B is found to be a more conductive host than Host A. 
The 50\% Host B mixture shows intermediate hole conductivity relative to pure materials. 
For electrons, at low voltage, the properties are similar to Host A, but transition to be Host B at high voltage. 
Thus, unlike conventional mixed EMLs where hole- and electron-transporting materials are blended to form a composite layer with overall improved charge transport, here Host B and the phosphorescent guest are likely responsible for both hole and electron transport. 
Thus, the addition of Host A to Host B to form the mixture favorably adjusts the transport properties of the EML for increased operational lifetime. 
Given the differences in architecture between the HODs and EODs, the relative mobility of electrons and holes cannot be quantitatively assessed. 
How these charge transport and injection properties translate to the exciton spatial profiles is most straightforwardly addressed by direct measurement of RZ.

\begin{wrapfigure}{r}{.5\textwidth}
\centering
\includegraphics[width=0.5\textwidth]{lifetimeApplications/dow_rz_spectra}
\caption{Raw spectral data for RZ measurements for hosts A, B and a 1:1 mixture}
\label{fig:dow_rz_spectra}
\end{wrapfigure}

The RZs of devices with 0\%, 50\%, and 100\% Host B were measured using a doped sensitizer approach, discussed in Chapter \ref{sec:rz_measurement}.\supercite{Bangsund2018}
The area-normalized exciton population map resulting from this measurement for each architecture is shown in Figure 3c (EL spectra from which these profiles are extracted are included in the supporting information). 
Host A is shown to have the highest exciton density of all three architectures, peaked at the ETL interface. 
Host B has the widest RZ, with a nearly flat exciton density across the EML. 
The 50\% Host B mixture shows an intermediate behavior with a narrower RZ than Host B that is also shifted away from the HTL/EML interface. 
This indicates that the EL lifetime improvement observed for the 50\% mixture compared to pure Host B cannot be attributed to a reduction in exciton density, as is the case in most mixed host EML OLEDs. 


\begin{wrapfigure}{r}{.5\textwidth}
\centering
\includegraphics[width=0.5\textwidth]{lifetimeApplications/dow_eqe_10}
\caption{\eqe for the 10 nm EML devices.}
\label{fig:dow_eqe_10}
\end{wrapfigure}
The photoluminescence degradation behavior shown in Figure \ref{fig:dow_lifetime}b becomes more stable with increasing composition of Host B, following the measured increase in RZ width decreased exciton density. 
This is consistent with our previous studies on the dependence of PL on RZ width in mixed host architectures and expectations based on exciton-induced-degradation kinetics.\supercite{Bangsund2018,Giebink2008a} 
Considering device degradation by bimolecular processes, devices with high concentration of Host A might be expected to exhibit a shorter lifetime than those based on Host B. 
However, the total degradation, shown in Figure \ref{fig:dow_lifetime}a shows a reduction in overall EL stability at concentrations above 30\% Host B, indicating a competing mechanism that does not affect the PL stability.  
Host B has the highest exciton population at the HTL interface, suggesting there may be an interfacial degradation process at play, creating a shorter lifetime.\supercite{Hershey2017,Wang2013} 
We hypothesize that as Host A is added to Host B, the RZ is shifted away from the HTL interface, causing less interfacial degradation and increasing the total lifetime, despite reductions in PL lifetime as the RZ narrows. 
However, the benefits of added Host A fall off rapidly below 15\% Host B as the injection barrier lowering provided by Host B are lost and significant narrowing of the RZ is observed.

\begin{wrapfigure}{r}{.5\textwidth}
\centering
\includegraphics[width=0.5\textwidth]{lifetimeApplications/dow_lifetime_10}
\caption{(a) EL and (b) PL lifetimes for the 10 nm EML devices.}
\label{fig:dow_lifetime_10}
\end{wrapfigure}

To minimize differences in RZ width and normalize the effect of the HTL interface on degradation, devices were fabricated with EMLs of only 10 nm, with efficiencies shown in Figure \ref{fig:dow_eqe_10}. 
This thin emissive layer ensures that excitons are present throughout the entire EML and RZ is minimally different between the devices. 
For these devices, the ETL thickness was set to 50 nm to center the electric field profile of the $\lambda$=473 nm pump laser in the EML. 
The peak external quantum efficiency of these devices was found to be 14 $\pm$ 1\% for all three architectures. 
The EL and PL lifetimes can be seen in Figure \ref{fig:dow_lifetime_10}a and b, respectively. The 50\% and 100\% Host B are found to have the identical lifetimes for both EL and PL,  = (22 $\pm$ 1) hr for EL, and  = (16 $\pm$ 1) hr for PL. 

This identical behavior in both EL and PL degradation suggests that the kinetics and mechanism of degradation are the same for devices based on Host B and the 50:50 mixture. 
Host A shows a shorter lifetime in both EL and PL, likely due to its narrow RZ which is heavily peaked at the ETL interface (Figure \ref{fig:dow_rz}c). 
Given this identical degradation behavior for a thin emissive layer, it is unlikely that the mixed host devices are longer lived due to enhanced morphological stability. 
Using a wide energy gap host along with an ambipolar host in this architecture has allowed tuning of the RZ away from the HTL interface. 
While successful in this device, it can be difficult to understand a priori how a wide energy gap material will shift the charge transport characteristics of the ambipolar host. 
This could make designing a co-host system around this type of material combination  a challenge. 
However, when investigating material systems where transport properties are unknown, this type of interaction may be encountered. 
This system is also an important counterexample to the assumption that reducing charge transport is always detrimental to device behavior. 
In fact, device lifetime can be a much more subtle and intricate design system than the simple maximization of RZ width. 

\subsection{Conclusion}
In this work, a unique co-host architecture is demonstrated using a wide energy gap host and an ambipolar host. 
Interestingly, despite a reduction in RZ width, an increase in device lifetime is observed. 
This increase in lifetime is demonstrated to be due to exciton formation at an unstable interface, known to accelerate degradation. 
This conclusion is contrary to the typical design rule that an increased RZ width is always beneficial to device lifetime. 
Rather, a balance of RZ width and position needs to be established, especially with regard to recombination at interfaces.
This result provides an example of the blend of the two mechanisms demonstrated in Sections \ref{sec:cbp_host} and \ref{sec:lifetime_meml}.


\ifcsdef{mainfile}{}{\printbibliography}
\end{document}
