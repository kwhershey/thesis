\documentclass[../thesis.tex]{subfiles}
\begin{document}
\section*{Abstract}

Over the last decade, organic light-emitting devices (OLEDs) have grown to receive tremendous attention for application in commercial displays and in lighting.  
In display applications, OLEDs offer a wide and tunable color gamut, high peak efficiency, high contrast, and compatibility with novel form factors.
For white-lighting, OLEDs can achieve broadband emission and are compatible with flexible substrates, allowing conformal lighting as well as potential for roll-to-roll, low cost processing.
OLED technology is ubiquitous for small format mobile displays, and is emerging in the large display and white lighting market.
This lag in commercial uptake is partially due to a lack of understanding of the underlying device operation.

This thesis seeks to refine the understanding of device electrical and optical operation.
The fundamental kinetics of charge carrier and light precursors (excitons) are investigated in a unified model for the transient and steady-state regimes of operation.
Through modeling, a quantification and understanding of performance metrics is developed, aiding in the characterization of OLEDs.
Optical modeling techniques were implemented to assist in the understanding of overall device behavior.

Operational stability is investigated through the development of a technique which separates the contributions of loss mechanisms during operation.
This is investigated in several systems, and enables a unique understanding of the interplay between distinct degradation mechanisms.
Molecular design was also considered via degradation of a molecular family within devices and in isolation.


\pagebreak
\end{document}

