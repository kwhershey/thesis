\documentclass[../thesis.tex]{subfiles}
\begin{document}
\section*{Abstract}

Over the last decade, organic light-emitting devices (OLEDs) have grown to receive tremendous attention for application in commercial displays and in lighting.  While mostly successful for small format displays, challenges still exist that limit their performance for broader applications.  Many of these limitations stem from a lack of understanding of charge and exciton dynamics and their impact on efficiency and stability.  In this presentation, we describe novel device characterization and modelling efforts aimed at elucidating key dynamic processes in multiple regimes, including the microsecond transient behavior, steady-state, and long term degradation.  

A model is presented which unifies both the transient and steady-state electroluminescence behavior of an OLED as a function of current density.  The excellent agreement between the model and experiment enables a deeper understanding of efficiency reduction at high brightness.  Additionally, the relatively ambiguous device efficiency parameter of charge balance is recast as an exciton formation efficiency.  This framework permits a novel characterization paradigm for decoupling degradation pathways during OLED life-testing.  In addition to the luminance loss, the degradation in emitter photoluminescence and exciton formation efficiency are also extracted.  This technique is applied to an archetypical phosphorescent OLEDs, enabling more comprehensive design rules for device engineering to realize enhanced lifetime.  Data science is a rising topic in industrial research.  A system for enabling data science techniques within laboratory research is presented.  Select useful applications are demonstrated.

\pagebreak
\end{document}

