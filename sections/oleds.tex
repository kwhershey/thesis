\documentclass[../thesis.tex]{subfiles}
\begin{document}
\chapter{Organic Light-Emtting Devices}\label{sec:oleds}

\section{Fabrication Processes}

Within our lab, the standard fabrication process for OLEDs is thermal evaporation at base pressures <10$^{-7}$ Torr.  Substrates consist of glass precoated with indium-tin oxite (ITO).  Prior to deposition, substrates are cleaned and treated in a UV-ozone environment.  Large area devices on patterned ITO are spin coated with a solution processed hole conducting planarizing layer.

\section{Characterization}
\subsection{Luminance}
\subsection{Efficiency Analysis}

\section{Historical Developement}
\subsection{The First OLEDs}
\subsection{Phosphorescence}
\subsection{Host-Guest Systems}
\subsection{Cohost Systems}
\subsection{Thermally Activated Delayed Fluorescence}

\section{Device Operation}
\subsection{Dynamic Processes}
\subsection{Efficiency Roll-Off}

\section{Recombination Zone Characterization}

\section{Single Carrier Devices}



\ifcsdef{mainfile}{}{\bibliography{../library}}
\end{document}
